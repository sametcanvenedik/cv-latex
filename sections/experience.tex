% --------- PROFESSIONAL & ACADEMIC EXPERIENCE ----------
& \section{MESLEKİ \& AKADEMİK DENEYİM}\\

& \textbf{Physmart Solutions,} Türkiye\\
2024 -- Günümüz  & Mekatronik Mühendisi    \\[-12pt]
\cvitem{
   \item Elektromekanik cihazların kavramsal tasarımdan seri üretim aşamasına kadar olan süreçleri SolidWorks kullanılarak gerçekleştirildi
    \item Prototipleme ve yapısal optimizasyon çalışmaları aracılığıyla Ar-Ge odaklı ürün geliştirme süreçlerine destek sağlandı
    \item Mevcut bir cihazın mekanik yeniden tasarım süreci kapsamında; gövde (enclosure) modellemesi ve dış üretim süreçlerinin koordinasyonu yürütüldü
    \item Yük kaynaklı deformasyonun (sehim) değerlendirilmesi amacıyla temel yapısal simülasyonlar gerçekleştirildi
    \item Mekanik montajlar, PCB tasarımları (Altium) ve STM32 tabanlı gömülü yazılımları kapsayan sistem seviyesinde hata ayıklama (debugging) çalışmaları yürütüldü
    \item Hibrit yerel/uzak veri yönetimini entegre edecek şekilde, PC tabanlı kontrol yazılımı Python ve Qt kullanılarak yeniden geliştirildi
}
\\[-6pt]

& \textbf{BTECH Innovation,} Türkiye\\
2022 -- 2023  & Staj    \\[-12pt]      
\cvitem{
   \item Eğitim ve değerlendirme parçaları üzerinde kafes (lattice) ve HEX yapı tasarımı için nTopology yazılımı bağımsız olarak öğrenildi ve uygulandı
    \item Eklemeli imalat çalışmaları kapsamında, gerilme odaklı tasarım (stress-informed design) konseptleri kullanılarak HEX tabanlı hafifletilmiş geometriler tasarlandı
    \item Üretim iş akışlarında aktif rol alınarak SLA, FDM ve SLS eklemeli imalat süreçlerinde pratik deneyim kazanıldı
    \item Eklemeli imalat sistemleri operasyonel olarak kullanıldı; destek stratejileri, üretim kısıtları ve malzeme davranışı konularında uygulamalı yetkinlik geliştirildi
    \item Üretim dışı bileşenler üzerinde, üretilebilirlik ve yapısal verimliliği değerlendirmek amacıyla eklemeli imalat için tasarım (DfAM) çalışmaları yürütüldü
}\\[-20pt]