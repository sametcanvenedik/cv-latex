% --------- PROJECTS ----------
& \section*{PROJELER} \\
2024--2025 & \textbf{Endüstriyel Otomasyon ve Ölçüm Sistemi (Gizli)}\\
& Şirket İçi Ar-Ge Projesi, Physmart Solutions
\cvitem{
    \item İç operasyonlardaki manuel iş gücünü azaltmak amacıyla tam otonom bir ölçüm ve veri işleme sistemi tasarlanmış ve geliştirilmiştir
    \item Mekanik tasarım, donanım seçimi, yazılım mimarisi ve sistem entegrasyonunu kapsayan uçtan uca sistem geliştirme sürecine liderlik edilmiştir
    \item Doğrusal mekanizmaların, anahtarların (switch) ve çevre birimlerinin entegrasyonu dahil olmak üzere; mekanik yapı ve Kartezyen hareket sistemi SolidWorks kullanılarak tasarlanmıştır
    \item Gerçek zamanlı cihaz kontrolü, veri toplama, işleme, görselleştirme ve otonom karar verme yeteneklerine sahip Python ve Qt tabanlı merkezi kontrol yazılımı geliştirilmiştir
    \item STM32 kontrollü alt sistemlerle USB tabanlı haberleşen çoklu işlem (multi-process) mimarisi, gerçek zamanlı sistem izleme, hata tespiti ve diyagnostik günlükleme (logging) özellikleri uygulanmış; sistem sürekli operasyonel kullanım ile doğrulanmıştır
}
\\
& \textbf{Yüksek Hassasiyetli Cherenkov Tabanlı Elektro-Optik Test ve Ölçüm Sistemi}\\
2024--2025 & TÜBİTAK 1507 Ar-Ge Projesi, Physmart Solutions
\cvitem{
    \item Radyoterapi cihaz hassasiyetini, Cherenkov radyasyon görüntüleri ile tedavi planlama verilerini karşılaştırarak doğrulamak amacıyla bir test ve ölçüm sistemi tasarlanmış ve geliştirilmiştir
    \item Mekanik tasarım, prototipleme, yazılım geliştirme ve deneysel doğrulama süreçlerini kapsayan uçtan uca sistem geliştirme sürecine liderlik edilmiştir
    \item SolidWorks kullanılarak hassas mekanik yapı (0,1 mm rotasyonel, 1 mm eksenel hassasiyet) tasarlanmış ve tasarım kararları temel sehim (deflection) analizleri ile desteklenmiştir
    \item OpenCV ve üçüncü parti kamera kütüphaneleri kullanılarak; gerçek zamanlı görüntü alma, işleme, görselleştirme ve veri günlükleme özelliklerine sahip Python ve Qt tabanlı bir PC uygulaması geliştirilmiştir
\item Sistem hassasiyeti; deneysel radyasyon ölçümleri ve yerleşik kalibrasyon metodolojileri ile yapılan karşılaştırmalı analizler sonucunda valide edilmiştir}
\\